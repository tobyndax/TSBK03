\documentclass[10pt]{article}

\usepackage{times}
\usepackage{mathptmx}
\usepackage{amsmath}
\usepackage{mathtools}
\usepackage{graphicx}
\usepackage{caption}
\usepackage{subcaption}
\usepackage{placeins}
\usepackage[inline]{enumitem}
\usepackage[margin=1.0in]{geometry}

\setlength\parindent{0pt}

\raggedbottom
\sloppy

\title{Computer Graphics\\
\emph{TSBK03}}

\author{David Habrman \\ davha227, 920908-2412\\
Jens Edhammer \\ jened502, 920128-5112 }

\date{\today}
\begin{document}

\maketitle

\newpage
\newpage

\section{Introduction}

This project aims was to implement Voronoi shattering. The two lists below
are from the project specification and shows which features had to be
implemented (will do) and which features were optional (might do).
The features have a tag, [done], if they were implemented.

\subsection{Features}

Will do:
\begin{itemize}
  \item Light model \hfill \\
  Implementation of Phong-shading. [done]
  \item Test area \hfill \\
  A simple floor that objects can fall on/rest on. [done]
  \item Gravitiy simulation \hfill \\
  Make object fall to the ground.
  \item Collision detection with ground \hfill \\
  Detect when objects hit the ground. [done]
  \item Collision detection between objects \hfill \\
  Detect when objects collide with each other. [done]
  \item Collision response with ground \hfill \\
  Make objects respond properly when hit the ground.
  \item Collision response between objects \hfill \\
  Make objects respond properly when colliding with each other.
  \item Skybox [done]
  \item Camera controls \hfill \\
  Keyboard and mouse movement controls (direction of camera). [done]
  \item Voronoi shattering for a simple 2D object \hfill \\
  Shatter a 2D square. [done]
  \item Voronoi shattering for a simple 2D wood object \hfill \\
  Shatter a 2D square and make it look like wooden splinters. [done]
  \item Voronoi shattering for a simple 3D object \hfill \\
  Shatter a 3D cube. [partially done?]
  \item Voronoi shattering for a simple 3D wood object \hfill \\
  Shatter a 3D cube and make it look like wooden splinters. [partially done?]

\end{itemize}



Might do:
\begin{itemize}

  \item Indoor test environment \hfill \\
  Create a room for visual effect.
  \item Indoor lighting
  \item Voronoi shattering for a simple 3D wood object \hfill \\
  Change distance metric to implement wood like splinters. [partially done?]
  \item Voronoi shattering for two simple 3D objects colliding \hfill \\
  One object may be ”harder” and will not break.
  \item Voronoi shattering for a complex object.
  \item Convex collision detection \hfill \\
  Collision detection for convex objects. [done]

\end{itemize}

\section{Background information}
\subsection{Voronoi shattering}
This is a non physical way of shattering objects. The idéa is to create a
seed pattern inside the object you want to shatter. This pattern is then used
to create fragments from the original object. The original object is split
in the middle of each seed point giving rise to one fragment for each
seed. The pattern can have different appearance depending on how you want your
fragments to look. A uniform seed pattern will give uniform fragments but if
you, for example, have a bullet hitting glass you want smaller fragments near
the impact point and larger fragments further away. This can be achieved by
choosing a non-uniform seed pattern with more and closer seeds near the impact
point and less sparser points further away.

\subsection{Collision detection}

\subsection{Collision response}


\FloatBarrier


\newpage
\section{About our implementation}
\subsection{Voronoi shattering}

\subsection{Collision detection}

\subsection{Collision response}

\section{Interesting problems}

\section{Limitations}

\section{Conclusion}

\end{document}
