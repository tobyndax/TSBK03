\documentclass[10pt]{article}

\usepackage{times}
\usepackage{mathptmx}
\usepackage{amsmath}
\usepackage{mathtools}
\usepackage{graphicx}
\usepackage{caption}
\usepackage{subcaption}
\usepackage{placeins}
\usepackage[inline]{enumitem}
\usepackage[margin=1.0in]{geometry}

\setlength\parindent{0pt}

\raggedbottom
\sloppy

\title{Computer Graphics\\
\emph{TSBK03}}

\author{David Habrman \\ davha227, 920908-2412\\
Jens Edhammer \\ jened502, 920128-5112 }

\date{\today}
\begin{document}

\maketitle

\newpage
\newpage

\section{Introduction}

This project aims was to implement Voronoi shattering. The two lists below
are from the project specification and shows which features had to be
implemented (will do) and which features were optional (might do).
The features have a tag, [done], if they were implemented.

\subsection{Features}

Will do:
\begin{itemize}
  \item Light model \hfill \\
  Implementation of Phong-shading. [done]
  \item Test area \hfill \\
  A simple floor that objects can fall on/rest on. [done]
  \item Gravitiy simulation \hfill \\
  Make object fall to the ground.
  \item Collision detection with ground \hfill \\
  Detect when objects hit the ground. [done]
  \item Collision detection between objects \hfill \\
  Detect when objects collide with each other. [done]
  \item Collision response with ground \hfill \\
  Make objects respond properly when hit the ground.
  \item Collision response between objects \hfill \\
  Make objects respond properly when colliding with each other.
  \item Skybox [done]
  \item Camera controls \hfill \\
  Keyboard and mouse movement controls (direction of camera). [done]
  \item Voronoi shattering for a simple 2D object \hfill \\
  Shatter a 2D square. [done]
  \item Voronoi shattering for a simple 2D wood object \hfill \\
  Shatter a 2D square and make it look like wooden splinters. [done]
  \item Voronoi shattering for a simple 3D object \hfill \\
  Shatter a 3D cube. [partially done?]
  \item Voronoi shattering for a simple 3D wood object \hfill \\
  Shatter a 3D cube and make it look like wooden splinters. [partially done?]

\end{itemize}



Might do:
\begin{itemize}

  \item Indoor test environment \hfill \\
  Create a room for visual effect.
  \item Indoor lighting
  \item Voronoi shattering for a simple 3D wood object \hfill \\
  Change distance metric to implement wood like splinters. [partially done?]
  \item Voronoi shattering for two simple 3D objects colliding \hfill \\
  One object may be ”harder” and will not break.
  \item Voronoi shattering for a complex object.
  \item Convex collision detection \hfill \\
  Collision detection for convex objects. [done]

\end{itemize}

\section{Background information}
\subsection{Voronoi shattering}
This is a non physical way of shattering objects. The idéa is to create a
seed pattern inside the object you want to shatter. This pattern is then used
to create fragments from the original object. The original object is split using
a form of nearest neighbor. Lines are drawn at the maximum distance from each
point giving rise to a pattern as in figure \ref{fig:normPattern}. The object
is then split along these lines giving rise to one fragment for each seed.
The pattern can have different appearance depending on how you want your
fragments to look. A uniform seed pattern will give uniform fragments but if
you, for example, have a bullet hitting glass you want smaller fragments near
the impact point and larger fragments further away. This can be achieved by
choosing a non-uniform seed pattern with more and closer seeds near the impact
point and less sparser points further away.

\begin{figure}
    \centering
    \includegraphics[width = 0.8\textwidth]{images/normPattern.jpg}
    \caption{Voronoi fragments using a random normal distributed pattern}
    \label{fig:normPattern}
\end{figure}

\subsubsection{Seed pattern}
As mentioned above the seed pattern can be a powerful tool to make objects
shatter the way you want. Voronoi shattering is not based on physics which is
why it's important to know how to make the fragments look the way you want.
A wooden object, for example, doesn't shatter the same way as a ceramic object
would. The basic implementation of Voronoi shattering is specially useful for
materials as ceramic and glass since it looks very realistic.

Let's shatter a 2D square. A uniform pattern would result in uniform fragments
as in figure \ref{fig:uniPattern}. This is probably not a pattern you would
want to use when shattering object. Let's say that the square is made out of
glass. Using a pattern with random normal distributed points as in figure
\ref{fig:normPattern} we get fragment suitable for shattering glass. To
illustrate the power of seed pattern further let's shatter a square made out
of wood. Wood shatters in splinters along the fiber direction. The seed pattern
can be used to give a square the illusion of having a vertical fiber direction.
Using a random pattern that is more sparse in the vertical direction and more
compact in the horizontal direction we get the pattern in figure \ref{fig:woodPattern}.
This pattern can be further manipulated to get longer more slim splinters.

Usually we also have a impact point. Fragment near the impact point are usually
smaller than fragments further away. As mentioned above this can be achieved
by choosing a non-uniform seed pattern with more and closer seeds near the impact
point and less sparser points further away.

\begin{figure}
    \centering
    \includegraphics[width = 0.8\textwidth]{images/uniPattern.jpg}
    \caption{Voronoi fragments using uniform pattern}
    \label{fig:uniPattern}
\end{figure}

\begin{figure}
    \centering
    \includegraphics[width = 0.8\textwidth]{images/woodPattern.jpg}
    \caption{Voronoi fragments using a pattern suitable for shattering wood}
    \label{fig:woodPattern}
\end{figure}


\subsubsection{Voronoi implementation}

\subsection{Collision detection}

\subsection{Collision response}


\FloatBarrier


\newpage
\section{About our implementation}
\subsection{Voronoi shattering}
Naive and slow implementation but it works.

\subsection{Collision detection}
Worked ok?

\subsection{Collision response}
Didn't work, why not?

\section{Interesting problems}
Collision response proved to be a difficult problem. The detection part was somewhat
simple. We managed to detect collision with the ground as well as collision with
other objects.

\section{Limitations}
Can't shatter object, only a square. Can't do 3D, we have 2,5D.

\section{Conclusion}

\end{document}
